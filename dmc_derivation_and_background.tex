\documentclass{article}
\usepackage{amsmath}
\usepackage[utf8]{inputenc}
\begin{document}
\title{Diffusion Monte Carlo Introduction and Derivation}
\author{Luke Hatcher}
\date{University of Washington: Department of Chemistry \\ McCoy Group, Fall 2019}

\maketitle

\section{Introduction to Diffusion Monte Carlo}
Diffusion Monte Carlo (DMC) is a numerical method for solving the time independent Schrodinger equation. 
At its core, DMC is computational algorithm. 
The algorithm approximates the zero point energy (ZPE) of a system as well as produces an accurate representation of $\Psi$.$^\text{[1]}$
\subsection{Applications of DMC}
DMC is a useful approach for studying the vibrations of molecules that exhibit anharmonic behavior. 
Molecules with lots of anharmonicity are poor candidates for other methods.  

Applications include investigating the effect of adding zero point energy to a system and observing what happens to the wavefunction when the system can access different parts of the potential. 
This can give insight into the structural configuration of the system. 

Furthermore, while the style of DMC detailed in this document limits results to the ground state vibrational energy, further more complicated applications include approximating higher order energies.$^\text{[2]}$

\section{Derivation}
The goal of this section is to motivate Diffusion Monte Carlo (DMC) by providing a well outlined derivation to the best of my understanding.
% ----------------------
\subsection{Imaginary Time}
Although DMC solves the TISE, it starts with the TDSE equation and time dependent quantum mechanics are applied to solve it.
\begin{equation} \label{eqn}
{i\hbar\frac{\partial}{\partial t}}\Psi(r,t)=\hat{H}{\Psi(r,t)}
\end{equation}

The solutions as a superposition of eigenstates of the Hamiltonian:

\begin{equation} \label{eqn}
\Psi(r,t)=\sum_{n}c_{n}e^{\frac{{-i}E_{n}t}{\hbar}}\psi(r)
\end{equation}

Complete a Wick rotation by substituting $\frac{it}{\hbar}=\tau$ to transform to imaginary time.

\begin{equation} \label{eqn}
\Psi(r,\tau)=\sum_{n}c_{n}e^{-E_{n}\tau}\psi(r)
\end{equation}
% -----------------------
\subsection{Discretize time}
In order for this algorithm to work, it is important to note that imaginary time $\tau$ must be discretized into time steps of length $\Delta\tau$. 
A smaller $\Delta\tau$ is more accurate however it is also more computationally expensive. 
The algorithm is ran for a finite number of time steps where at any point, $\tau$ is equivalent to the corresponding timestep.

\subsection{Shift energy by $V_{ref}$}
For reasons that will become more apparent later, we shift the energy in equation (3) by  a reference amount $V_{ref}$:
\begin{equation} \label{eqn}
V_{ref}(\tau)=\overline{V}(\tau)-\alpha\frac{N_{w}(\tau)-N_{w}(\tau_{o})}{N_{w}(\tau_{o})},\;\alpha=\frac{1}{2\Delta\tau}
\end{equation}

Plugging this into (3) results in:

\begin{equation} \label{eqn}
\Psi(r,\tau)=\sum_{n}c_{n}e^{-(E_{n}-V_{ref})\tau}\psi(r)
\end{equation}

Here, $V_{ref}$ is composed of several parts. 
For now, this equation relies on several details of the algorithm which are explained later in section 3. 
For now all that needs to be understood is that $N_{w}$ is the number of "walkers" at that point and $\overline{V}(\tau)$ is the average of each "walkers" potential at time $\tau$. 
The importance of $V_{ref}$ is shown next in section 2.4 and its application to the algorithm will be outlined in section 3 as well. 

\subsection{ZPE convergence}
The convenience of DMC is that when $E_{0}=V_{ref}$, 
\begin{equation} \label{eqn}
\Psi(r,\tau)=\sum_{n}c_{n}e^{-(E_{n}-V_{ref})\tau}\psi(r) = c_{0}\psi_{0}
\end{equation}

Thus assuming that for long enough time $\tau$, $V_{ref}$ converges to $E_{0}$:

\begin{equation} \label{eqn}
\lim_{\tau\to\infty}\sum_{n}c_{n}e^{-(E_{n}-V_{ref})\tau}\psi(r) = c_{0}\psi_{0}
\end{equation}

Therefore the ZPE can be found upon this convergence.
Now in order to do so, there needs to be a way to propagate forward in imaginary time. 

\subsection{Propagation of $\tau$}
Since the Hamiltonian is time independent, let's consider the time evolution operator:
\begin{equation} \label{eqn}
\hat{U}=e^{\frac{-iEt}{\hbar}}
\end{equation}

Including the Wick rotation from section 2.1:
\begin{equation} \label{eqn}
\hat{U}=e^{-\hat{H}\tau}
\end{equation}
Normally, $\hat{U}$ as written would now be operated on (6) in order to propagate forward. 
However an approximation will be used to simplify the operator to in turn simpligy the algorithm. 
Recalling that $\hat{H}=\hat{T}+\hat{V}$ and Taylor expanding $\hat{U}$ results in:

\begin{equation} \label{eqn}
e^{-\hat{H}\Delta\tau}=1+(\hat{T}+\hat{V})\Delta\tau-\frac{1}{2}(\hat{T}+\hat{V})^2\Delta\tau^2+...
\end{equation}

recombining terms:
\begin{equation} \label{eqn}
=(1-\hat{T}\Delta\tau+\frac{1}{2}\hat{T}^2\Delta\tau^2-...)(1-\hat{V}\Delta\tau+\frac{1}{2}\hat{V}^2\Delta\tau^2-...)+(\hat{T}\hat{V}-\hat{V}\hat{T})\Delta\tau^2+...
\end{equation}

which is approximately equivalent to: 
\begin{equation} \label{eqn}
\approx{e^{-\hat{T}\Delta\tau}e^{-\hat{V}\Delta\tau}+[\hat{T},\hat{V}]\Delta\tau^2+...}
\end{equation}

here the commutator of $\hat{T}$ and $\hat{V}$ is viewed as a correction value. Due to the $\Delta\tau^2$ term and considering a small enough value of $\Delta\tau$, this term is omitted in the approximation, leaving:

\begin{equation} \label{eqn}
\approx{e^{-\hat{T}\Delta\tau}e^{-\hat{V}\Delta\tau}}
\end{equation}

furthermore, adding in $V_{ref}$ from section 2.3 results in:
\begin{equation} \label{eqn}
\approx{e^{-\hat{T}\Delta\tau}e^{-(\hat{V}-V_{ref})\Delta\tau}}
\end{equation}

\subsection{Representing $\Psi$}
In Diffusion Monte Carlo, an ensemble of localized functions is used to represent $\Psi$.
\begin{equation} \label{eqn}
\Psi(r,\tau)=\sum_{i}f(r-r_{i}(\tau))
\end{equation}

here, the sum is indexed by the number of walkers at that point and each walker only has amplitude at position $r_{i}$ at time $\tau$. 

\subsection{Operating on $\Psi$}
In order to propagate forward, we will operate our time evolution operator approximation (14) on the representation of $\Psi$ (15).  

\begin{equation} \label{eqn}
\Psi(r,\tau+\Delta\tau)=e^{-(\hat{V}-V_{ref})\Delta\tau}e^{-\hat{T}\Delta\tau}\sum_{i}f(r-r_{i}(\tau))
\end{equation}

The operator (14) is composed of a kinetic and a potential portion. 
It does not matter which portion operates on $\Psi$ first. 
However, for algorithmic simplicity, the kinetic portion will act first. 
Note that in the following equations the summation from (16) has been omitted. 
This is done for simplicity where only one walker $i$ is being worked with.

\begin{equation} \label{eqn}
e^{-\hat{T}\Delta\tau}f(r-r_{i}(\tau))
\end{equation}
This operation involves a Fourier transform from position to momentum space and back in order to solve. 
The details of this are beyond the scope of this paper, however the result is a Gaussian function of width $\sqrt{\frac{\Delta\tau}{m}}$. 
The operation of the potential energy follows suit:

\begin{equation} \label{eqn}
e^{-(\hat{V}{((r_{i}(\tau+\Delta\tau))}-V_{ref})\Delta\tau}f(r-r_{i}(\tau+\Delta\tau))
\end{equation}

here, the walker $i$ located at $r_{i}$ at time $\tau+\Delta\tau$ has its potential energy evaluated at that point. 
Furthermore, the reason $\tau+\Delta\tau$ is used here rather than $\tau$ is because of the order we operated the two portions of the time evolution operator (16). 
The Gaussian made in (17) at time $\tau$ is not used until time $\tau+\Delta\tau$. 

Much of this may be difficult to understand without a better background on the actual algorithm which is being used which will now be explained.

\section{Algorithm}
The first thing to understand is what exactly a "walker" is. 
Mathematically, the walkers are localized functions. 
Algorithmically, a walker is a set of coordinates that correspond to the molecule the DMC simulation is being run on. For example, if DMC is being ran on H$_{2}$O, one walker would be composed of two Hydrogen coordinates and one Oxygen coordinate for a total of three coordinates.

Prior to starting the simulation, an amount of initial walkers are chosen and each walker is given the equilibrium coordinates of the molecule of interest. 
\subsection{Each timestep}
Displace each walker by amount drawn randomly from Gaussian of width $\sqrt{\frac{\Delta\tau}{m}}$. 
Calculate $V_{ref}$ from prior timestep using equation 4. 
Evaluate each walker at the potential $V(r_{i})$. 
Now using each walkers evaluated potential, it is compared to the $V_{ref}$. 
Each walker is then assigned a probability of either replicating or deleting itself based on criteria listed below.


\begin{equation}
P_b = \begin{cases}
e^{(-(V(r_{i})-V_{ref})\Delta\tau)-1} &\text{if $V(r_{i})<V_{ref}$}\\
0 &\text{if $V(r_{i})>V_{ref}$}
\end{cases}
\end{equation}

\begin{equation}
P_d = \begin{cases}
1-e^{(-(V(r_{i})-V_{ref})\Delta\tau)} &\text{if $V(r_{i})>V_{ref}$}\\
0 &\text{if $V(r_{i})<V_{ref}$}
\end{cases}
\end{equation}


Once probabilities of birth or death have been assigned to a walker, a uniformly random number $X$ is drawn on the interval $[0, 1]$. 
This number is compared to the walkers $P_{b}$ or $P_{d}$. If $X$ is less than $P_{b}$ or $P_{d}$, the walker is birthed or killed respectively. This replication and deletion process leads to $V_{ref}$ converging to $E_{0}$. 

\subsection{Extracting ZPE results}
As detailed in section 2.4, for long enough $\tau$, $V_{ref}$ converges to $E_{0}$. Thus, once the DMC simulation has ran to completion, taking the average of each $V_{ref}$ collected at each timestep is approximately equal to the ZPE. 
However due to the nature of convergence, the first $10-20\%$ of the simulations $V_{ref}$ values are not used in the average calculation to account for the time the ensemble took to converge. 


\subsection{Extracting $\Psi^2$ using descendant weighting}
Since we are representing $\Psi$ as an ensemble of localized functions(walkers), then the walker coordinates at the end of the simulation is equal to the wavefunction $\Psi$. 
Given the fact that $\Psi^2$ is a probability density, by assigning weights to each walker, a probability density is able able to be plotted. 

Returning to the replication and deletion portion of the algorithm, each time a walker is replicated it gets a weight of $+1$ and each time it is deleted it gets a weight of $0$. 
Collecting this data allows for $\Psi^2$ to be projected onto the bond length by histogramming along the bond length and using the walkers weights as their level of contribution to each histogram bin. 
The result is a probability density representing $\Psi^2$.

\section{References}
[1] James B. Anderson, Quantum chemistry by random walk. J. Chem. Phys. 65, 4121 (1976); https://doi.org/10.1063/1.432868\\

[2] Anne B. McCoy, Eric G. Diken, and Mark A. Johnson, Generating Spectra from Ground-State Wave Functions: Unraveling Anharmonic Effects in the OH-H$_{2}$O Vibrational Predissociation Spectrum, The Journal of Physical Chemistry A 2009 113 (26), 7346-7352
DOI: 10.1021/jp811352c




\end{document}
